% Options for packages loaded elsewhere
\PassOptionsToPackage{unicode}{hyperref}
\PassOptionsToPackage{hyphens}{url}
%
\documentclass[
]{article}
\usepackage{amsmath,amssymb}
\usepackage{iftex}
\ifPDFTeX
  \usepackage[T1]{fontenc}
  \usepackage[utf8]{inputenc}
  \usepackage{textcomp} % provide euro and other symbols
\else % if luatex or xetex
  \usepackage{unicode-math} % this also loads fontspec
  \defaultfontfeatures{Scale=MatchLowercase}
  \defaultfontfeatures[\rmfamily]{Ligatures=TeX,Scale=1}
\fi
\usepackage{lmodern}
\ifPDFTeX\else
  % xetex/luatex font selection
\fi
% Use upquote if available, for straight quotes in verbatim environments
\IfFileExists{upquote.sty}{\usepackage{upquote}}{}
\IfFileExists{microtype.sty}{% use microtype if available
  \usepackage[]{microtype}
  \UseMicrotypeSet[protrusion]{basicmath} % disable protrusion for tt fonts
}{}
\makeatletter
\@ifundefined{KOMAClassName}{% if non-KOMA class
  \IfFileExists{parskip.sty}{%
    \usepackage{parskip}
  }{% else
    \setlength{\parindent}{0pt}
    \setlength{\parskip}{6pt plus 2pt minus 1pt}}
}{% if KOMA class
  \KOMAoptions{parskip=half}}
\makeatother
\usepackage{xcolor}
\usepackage[margin=1in]{geometry}
\usepackage{graphicx}
\makeatletter
\def\maxwidth{\ifdim\Gin@nat@width>\linewidth\linewidth\else\Gin@nat@width\fi}
\def\maxheight{\ifdim\Gin@nat@height>\textheight\textheight\else\Gin@nat@height\fi}
\makeatother
% Scale images if necessary, so that they will not overflow the page
% margins by default, and it is still possible to overwrite the defaults
% using explicit options in \includegraphics[width, height, ...]{}
\setkeys{Gin}{width=\maxwidth,height=\maxheight,keepaspectratio}
% Set default figure placement to htbp
\makeatletter
\def\fps@figure{htbp}
\makeatother
\setlength{\emergencystretch}{3em} % prevent overfull lines
\providecommand{\tightlist}{%
  \setlength{\itemsep}{0pt}\setlength{\parskip}{0pt}}
\setcounter{secnumdepth}{5}
\ifLuaTeX
  \usepackage{selnolig}  % disable illegal ligatures
\fi
\usepackage{bookmark}
\IfFileExists{xurl.sty}{\usepackage{xurl}}{} % add URL line breaks if available
\urlstyle{same}
\hypersetup{
  pdftitle={Project Proposal: Happiness Report's top tier for multiple consecutive years},
  pdfauthor={Amal Dagan, Bar Evron, Niv Dolev, Roni London},
  hidelinks,
  pdfcreator={LaTeX via pandoc}}

\title{Project Proposal: Happiness Report's top tier for multiple
consecutive years}
\author{Amal Dagan, Bar Evron, Niv Dolev, Roni London}
\date{}

\begin{document}
\maketitle

\newpage

\section{Introduction}\label{introduction}

\textbf{Research Question}\\
Which indicators enable a country to remain in the World Happiness
Report's top tier for multiple consecutive years?

\textbf{Background}\\
GDP per capita, social support, life expectancy and perceived freedom
are repeatedly cited as key drivers of well-being.

\textbf{Current Gap and existing research}\\
Several studies have examined national happiness using WHR data, though
few have explored long-term patterns. For example, Kuppens et al.~(2023)
analyzed cultural factors in 78 countries before and after the COVID-19
pandemic. They found that individualism and indulgence became stronger
predictors of happiness post-COVID, suggesting that global crises may
shift the importance of cultural values. However, their analysis
compared only two time points (2017--2019 vs.~2021) and did not assess
whether these associations hold across a longer period. Similarly,
Al‐Maatouq and Al Shammari (2022) examined the relationship between WHR
scores, Hofstede's cultural dimensions, and government education
spending across 58 countries. They found that countries with higher
long-term orientation and lower power distance tended to score higher on
the WHR, indicating possible structural links between culture and
well-being. However, their study was cross-sectional and did not account
for how such associations might change over time. Despite the value of
these contributions, our research takes a unique direction. We combine
an unusually broad dataset---spanning 2011 to 2024 and covering over 150
countries---with repeated-measures modeling to assess which happiness
predictors remain robust over time. Unlike studies that focus on a
single year or a narrow method, our approach allows us to explore both
consistency and change in global well-being, providing updated and
policy-relevant insights.

\textbf{Method \& ``Stability Score''}\\
A longitudinal logistic model is re-estimated for each vintage; a
variable earns a stability score when its effect remains significant in
all fourteen years.

\textbf{Contribution}\\
Turning the dataset's continual renewal from hurdle to asset, we deliver
up-to-date policy guidance and reveal whether the perennial ``Top-20''
countries stay happy for the same reasons or for shifting,
time-dependent ones.

\section{Data}\label{data}

We use publicly available data from the World Happiness Report, based on
the Gallup World Poll, covering the years 2011 to 2023. The dataset
contains 1,969 yearly observations across 125 to 153 countries per year,
with 14 variables per country-year entry. Each observation represents a
single country in a given year. The variables include social, health,
and economic factors widely studied in happiness research, such as:
social support, GDP per capita, healthy life expectancy, freedom to make
life choices, generosity, and perceptions of corruption. Our binary
outcome variable indicates whether a country was ranked in the Top 20
happiest countries that year. Importantly, the happiness score itself is
not a single-year estimate but a three-year rolling average, meaning
each year's score reflects data from the current and two previous years.
For example, the 2022 score is an average of survey results from 2020,
2021, and 2022. This design introduces temporal dependence between
consecutive years. To address this, we plan to either: use
repeated-measures models (e.g., mixed-effects models) that account for
within-country correlations across time, or aggregate data at the
country level to study long-term patterns, such as overall average
happiness or total number of Top 20 appearances.

A detailed data dictionary is included in the /data/README.md file in
our repository.

\section{Preliminary Results}\label{preliminary-results}

\includegraphics{project_files/figure-latex/top_countries_long-1.pdf}

The distribution of the number of years in the top 20 is kept pretty
limited, only 14 countries have never left the top 20 spot. This goes to
show that being a top country is a perpetual title. Which means there is
merit to asking this question, we think there is a pattern of data that
indicates how consistent a country is at staying at the top of the list.

\#\texttt{\{r\ plot\_Ladder\_score\ ,fig.width=15,\ fig.height=5,\ echo=FALSE,\ warning=FALSE,\ message=FALSE\}\ \#plt1\ \#}

\#This graph's purpose is to show that the columns are good predictors
for the ladder score, and by extension being a top country. This wasn't
unexpected, it's logical that an indicator of a positive trait in a
country's society would impact its overall happiness. And since the
ladder score is what dictates where a country is ranked, it will also
probably take part in the model for our question.

\section{Work Plan}\label{work-plan}

The outcome is the Top-20 column and the predictors currently are all of
the columns. The way we will answer the question is that we will attempt
to find the most contributing variables in order to answer the question.

we will use a variety of methods to answer the question: SHAP
variables,desicion-tree and a logistic regression model, we will attempt
to build those models concurrently and to compare and contrast them.

we would like to receive some conclusive results(the p\_value of some of
the variables will be significant),in all models so after comparison we
could isolate the most relevant factors.

\begin{itemize}
\tightlist
\item
  Data preparation and cleaning. (Bar)
\item
  Exploratory analysis and visualization. (Roni)
\item
  Model building and evaluation. (Roni,Niv)
\item
  Interpretation of results and final reporting. (Amal,Niv)
\end{itemize}

\newpage

Appendix\newline   This Markdown file describes the data structure and
organization for the project.\newline  

we will also elaborate about the pre project data cleaning,\newline  
this data is a addition of WHR(world happiness reports) from 2011 to
2024.\newline   the data cleaning included fixing the country
names(where the standard of naming was changed in between the
reports),erasing countries that dont exist and deleting
duplicates.\newline   we also changed the column names(not all of the
column names were identical in all the reports.)\newline  

Data Columns: \newline  
1. YEAR: the year in which the survey was conducted (the data is from
2011--2024)\newline  
2. Rank -- the country's Rank in the according year (from 1 to 158); not
all countries were ranked plus the rank is by 3 year average.\newline  
3. Country name -- the abbreviated name of the relevant
country.\newline  
4. Ladder score -- the WHR provides a number of the final happiness
score of the current country (from 1.3 to 7.9).\newline  
5. upperwhisker -- the upper boundary of the whisker score (from 1.43 to
7.91).\newline  
6. lowerwhisker -- the lower boundary of the whisker score (from 1.3 to
7.78).\newline  
7. Log GDP per capita -- a normalized version of the country's GDP index
in comparison to the rest of the world (normalized in range from 0 to
2.2).\newline  
8. Social support -- a normalized version of the country's social
support index in comparison to the rest of the world (normalized in
range from 0 to 1.8).\newline  
9. Healthy life expectancy -- a normalized version of the country's
healthy life expectancy index in comparison to the rest of the world
(normalized in range from 0 to 1.13).\newline  
10. Freedom to make life choices -- a normalized version of the
country's freedom to make life choices index in comparison to the rest
of the world (normalized in range from 0 to 1.018).\newline  
11. Generosity -- a normalized version of the country's generosity index
in comparison to the rest of the world (normalized in range from 0 to
0.57).\newline  
12. Perceptions of corruption -- a normalized version of the country's
perceptions of corruption index in comparison to the rest of the world
(normalized in range from 0 to 0.587).\newline  
13. Dystopia + residual -- a normalized version of the country's
dystopia + residual index in comparison to the rest of the world
(normalized in range from -0.11 to 3.482).\newline  
14. Top 20 -- a binary variable indicating whether the country is in the
top 20 in that particular year (the outcome variable).\newline  

Rows: 1,969\newline   Columns: 14\newline   \$ Year 2024, 2023, 2022,
2021, 2020, 2019\ldots{}\newline   \$ Rank 1, 143, 137, 146, 150, 153,
154,\ldots{}\newline   \$ Country\_name ``Finland'', ``Afghanistan'',
``Afghanistan'', \ldots{}\newline   \$ Ladder\_score 7.7360, 1.7210,
1.8590, 2.4040, \ldots{}\newline   \$ upperwhisker 7.810000, 1.775000,
1.923000, 2.\ldots{}\newline   \$ lowerwhisker 7.662000, 1.667000,
1.795000, 2.\ldots{}\newline   \$ Log\_GDP 1.7490000, 0.6280000,
0.6450000,\ldots{}\newline   \$ Social\_support 1.7830000, 0.0000000,
0.0000000,\ldots{}\newline   \$ Life\_expectancy 0.8240000, 0.2420000,
0.0870000,\ldots{}\newline   \$ Freedom\_of\_choices 0.9860000,
0.0000000, 0.0000000,\ldots{}\newline   \$ Generosity 0.1100000,
0.0910000, 0.0930000,\ldots{}\newline   \$ Corruption 0.502000000,
0.088000000, 0.0590\ldots{}\newline \$ Dystopia\_and\_residual 1.782000,
0.672000, 0.976000, 1.\ldots{}\newline   \$ Top\_20 TRUE, FALSE, FALSE,
FALSE, FALSE\ldots{}\newline  

\textbackslash{}\\
\textbf{models and fun} The main model: Decision Tree Like logistic
regression this is also a popular model for classification problems, and
since it can be visualized it is an obvious choice to be one of the
models. The second model example: Logistic Regression If we're to take a
classification route, then logistic regression would be a good model for
that. We got good results, but again maybe too good. All of the
predictors seem to be strong predictors which was good for the sake of
showing that there is a correlation, but the performance was too
accurate to use as the main model.

\begin{verbatim}
## Top_20 ~ Log_GDP + Social_support + Life_expectancy + Freedom_of_choices + 
##     Generosity + Corruption + Dystopia_and_residual
\end{verbatim}

\begin{verbatim}
## Confusion Matrix and Statistics
## 
##           Reference
## Prediction   0   1
##          0 345   4
##          1   5  45
##                                           
##                Accuracy : 0.9774          
##                  95% CI : (0.9576, 0.9896)
##     No Information Rate : 0.8772          
##     P-Value [Acc > NIR] : 3.132e-13       
##                                           
##                   Kappa : 0.8962          
##                                           
##  Mcnemar's Test P-Value : 1               
##                                           
##             Sensitivity : 0.9857          
##             Specificity : 0.9184          
##          Pos Pred Value : 0.9885          
##          Neg Pred Value : 0.9000          
##              Prevalence : 0.8772          
##          Detection Rate : 0.8647          
##    Detection Prevalence : 0.8747          
##       Balanced Accuracy : 0.9520          
##                                           
##        'Positive' Class : 0               
## 
\end{verbatim}

\end{document}
